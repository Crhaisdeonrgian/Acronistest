\documentclass[a4paper,12pt]{article}
\usepackage[utf8x]{inputenc}
\usepackage[T2A]{fontenc}
\usepackage[english,russian]{babel}
\usepackage{amsmath,amsfonts,amssymb,amsthm,mathtools}
\usepackage{graphicx}
\usepackage[left=2cm,right=2cm,
top=2cm,bottom=2cm,bindingoffset=0cm]{geometry}
\usepackage{amsfonts}
\begin{document}
	
\title{Задача для собеседования на кафедру}
	
\author{Игорь Вожга Б05-872 }
\date{Май 2020}
\maketitle
\newpage
\title{ \bf{Задача TSP}}

Формулировка: Дана карта, на которой отмечено $N$ поселений. Поселения находятся на различном расстоянии друг от друга. Торговец отправляется из поселения $N_0$. Предложите алгоритм, который позволит найти оптимальный маршрут для обхода всех $N$ поселений и вернуться в точку старта.

Ответ должен содержать один или несколько алгоритмов (можно псевдокод) и пояснение о эффективности данных решений.

Решение:

Для решения задачи будем использовать алгоритм ближайшего соседа и алгоритм $2-opt$, реализующий  $2$-оптимальную эвристику.

Для начала примитивынй алгоритм ближайшего соседа, устройство которого следует из названия, построит цикл, который будет близок к тому, к чему мы стремимся. После этого полученный nearest neighbour цикл я улучшу спомощью $2opt$.

$2$-оптимальная эвристика основана на построении окрестности для данного тура $\tau$. То есть для $\tau$ строится множество туров $\tau'$ полученных из $\tau$ спомощью удаления двух рёбер $(a,b)$ и $(c, d)$ и добавления рёбер $(a, c)$ и $(b, d)$. 

А алгоритм $2-opt$  являеятся реализацией данной эвристики для задачи $TSP$. Построим псевдокод:

$In:$ Множество поселений $P =\{p_1, \dots, p_N\} $; Координаты поселений $(x_i, y_i)$; 

Первоначальный тур: $\tau = (p_{0_1}, \dots , p_{0_n})$

$Out:$ Тур $T=(p'_1, \dots, p'_n)$

$begin:$
Обозн $Pairs_0=\{(i,j)| i, j \in {1,\dots,n} ;i\neq j \} $ - множество пар поселений

$Pairs = Pairs_0$

$repeat$

$Pairs - (a, b)$

$t' = \{p_1, \dots, p_{a-1}, p_{b}, p_{b-1} \dots , p_{a+1}, p_a, p_{b+1} \dots\}$

$if$ $length(\tau')<length(\tau)$ $then$ - где $lentgh$ - длина тура

$begin:$

$\tau = \tau'$

$Pairs = Pairs_0$

$end;$

$until$ $|Pairs|=0$ - пока не переберём все пары для текущего тура

$return$ $\tau'$

$end;$ 

Оптимальность алгоритма: 

Для начала докажем, что если для входа $x$ алгоритм $2-opt$  нашёл тур $T=(p_1, \dots, p_n)$, 

то количество $q$ пар $(a,b)$ поселений для которых 

$\rho(a,b)>\frac{2 l^*(x)}{\sqrt{i}}$ $\forall i \in 1, \dots, n \rightarrow$ $q<i$ (1)

(где $l^*(x)$ - оптимальная длина цикла являющегося решением для данного входа $x$)



Допустим, что это не так, тогда $q\geq i$. Рассмотрим пары поселений $(a, b)$ удовлетворяющие условию (1).

Покажем, что число конечных точек $b$ ограниченно.

Рассмотрим окружность радиуса $\frac{ l^*(x)}{\sqrt{i}}$ и предположим, что точки $b_1, \dots b_s$ лежат в данной окружности. Причём $s\geq\sqrt{i}$. А точки $a_1, \dots a_s$ - это соответствующие им начальные. 

Тогда по предположению расстояние между $b_1, \dots b_s$ не больше $\frac{2 l^*(x)}{\sqrt{i}}$. А из этого будет следовать, что между начальными $a_1, \dots a_s$ расстояние не меньше $\frac{2 l^*(x)}{\sqrt{i}}$, потому что иначе удалив рёбра $(a_1, b_1)$ и $(a_1, b_1)$ и вставив два других: $(a_1, a_2)$ и $(b_1, b_2)$ B $T$ получили бы более короткий тур $Т'$, но это противоречит факту, что $T$ - локально оптимальный тур полученный алгоритмом $2-opt$. Поэтому мы доказали, что существует $s>\sqrt{i}$ точек на расстоянии не меньше  $\frac{2 l^*(x)}{\sqrt{i}}$. А значит оптимальный цикл на поселениях $a_1, \dots a_s$ имеет длину не меньше чем $2l^*(x)$. Однако из неравенства треугольника мы знаем, что если $a_1, \dots a_s$ является подмножеством входного множества поселений $P$, то тк добавление точек не может уменьшить длину тура, то длина тура на точках $a_1, \dots a_s$ должна быть меньше $l^*(x)$. А значит мы пришли к противоречию. То есть внутри такой окружности может лежать $s\leq\sqrt{i}$ конечных точек $b_1, \dots b_s$.

Теперь покажем, что когда $q\geq i$ мы можем построить такое множество $P'$, что $|P'|\geq i$ и любые его точки лежат на расстоянии как минимум $\frac{ l^*(x)}{\sqrt{i}}$.

Для этого мы возьмём множество всех поселений $P$ и будем добавлять в $P'$ произвольные конечные точки $b_i$ из $P$ и удалять из $P$ все точки находящиеся в окрестности радиусом $\frac{ l^*(x)}{\sqrt{i}}$. Таким образом, исходя из предыдущего пункта так как мы могли удалить $s\leq\sqrt{i}$ конечных точек, то в построенном множестве $P'$ не меньше $\sqrt{i}$ точек.

Таким образом, в $P'$ расстояние между точками $>\frac{ l^*(x)}{\sqrt{i}}$, точек $>\sqrt{i}$ значит длина цикла на таком множестве $P'$ $l>l^*$, что противоречит неравенству треугольника, тк $P'$ - подмнво $P$. Таким образом мы доказали утверждение, что количество $q$ пар $(a,b)$ поселений для которых 

$\rho(a,b)>\frac{2 l^*(x)}{\sqrt{i}}$ $\forall i \in 1, \dots, n \rightarrow$ $q<i$

Теперь, если просуммировать по $i$ для $n$ точек и вместо рёбер уже рассматривать полный цикл, то получим выражение, определяющее оптимальность $2-opt$:

$$L^{2opt}\leq 4\cdot OPT\sqrt{n}$$(где $OPT$ - оптимальный размер цикла, $L^{2opt}$ - размер цикла, который вернул алгоритм $2opt$)



\end{document}